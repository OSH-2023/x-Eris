\documentclass[a4paper]{article}
\usepackage[text={165mm,245mm}]{geometry}
\usepackage{graphicx}
\usepackage{subfigure}
\usepackage{ctex}
\usepackage{float} 
\usepackage{listings}
\usepackage{xcolor}
\usepackage{amsmath}
\definecolor{mygreen}{rgb}{0,0.6,0}  
\definecolor{mygray}{rgb}{0.5,0.5,0.5}  
\definecolor{mymauve}{rgb}{0.58,0,0.82}  
  

\title{x-Eris 调研报告}
\author{小组成员:胡天羽,罗胤玻,李润时,万辰希,吴书让}
\begin{document}
\maketitle

\section{项目概述}
\section{项目背景}
\subsection{嵌入式操作系统}
\subsubsection{概述}
\subsubsection{小结}

\subsection{非虚拟文件系统}
\subsubsection{概述}
\subsubsection{小结}

\subsection{虚拟文件系统 - Linux}
\subsubsection{概述}
\subsubsection{小结}

\subsection{虚拟文件系统 - 嵌入式}
\subsubsection{概述}
本篇报告主要介绍了FreeRTOS-Plus-FAT以及RT-Thread中的文件系统实现细节。

在早期的嵌入式系统中,需要存储的数据比较少,数据类型也比较单一,往往使用直接在存储设备中的指定地址写入数据的方法来存储数据。然而随着嵌入式设备功能的发展,需要存储的数据越来越多,也越来越复杂,这时仍使用旧方法来存储并管理数据就变得非常繁琐困难。
因此需要新的数据管理方式来简化存储数据的组织形式,需要新的文件系统。
\subsubsection{FreeRTOS-Plus—FAT}
\begin{itemize}
\item 主要
\begin{itemize}
    \item \texttt{ff\_dir}: 用于访问文件夹中内容
    \item \texttt{ff\_fat}: 用于访问 FAT 文件系统
    \item \texttt{ff\_file}: 用于文件读写
    \item \texttt{ff\_ioman}: 管理缓存和挂载读写对象(介质)
    \item \texttt{ff\_format}: 格式化或分区介质
    \item \texttt{ff\_locking}: 加锁?
    \item \texttt{ff\_memory}: 从内存读取数据
    \item \texttt{ff\_stdio}: 用于文件管理(统计),相对路径转换
    \item \texttt{ff\_sys}: 用于映射文件系统到根目录
\end{itemize}
\item 辅助
\begin{itemize}
    \item \texttt{ff\_headers}: 管理所有头文件
    \item \texttt{ff\_time}: 获取时间
    \item \texttt{ff\_error}:  用于错误处理
    \item \texttt{ff\_crc}: 用于计算 CRC(循环检验码)
    \item \texttt{ff\_string}: 字符串库
\end{itemize}
\item 驱动
\begin{itemize}
    \item 各处理器相应的文件系统驱动
\end{itemize}
\end{itemize}

FreeRTOS-Plus-FAT 文件系统的标准 API 与标准 C 库使用相同的 errno 值。

标准 C 库中的文件相关函数返回 0 表示通过,返回 -1 则表示失败。如果返回 -1,则失败的原因 存储在名为 errno 的变量中,须单独检查。 同样,FreeRTOS-Plus-FAT 的标准 API 返回 0 表示通过,返回 -1 则表示失败, 该 API 还会针对各项 RTOS 任务维护 errno 变量。
\subsubsection{RT-Thread}
\subsubsection{小结}

\subsection{文件系统开发}
\subsubsection{概述}
\subsubsection{小结}

\section{立项依据}

%\begin{figure}[H]
%    \centering
%    \includegraphics[scale=0.37]{p1.jpg}
%    \caption{图1}
%\end{figure}


\section{前瞻性/重要性分析}
\subsection{嵌入式文件系统的必要性}
\subsection{...}
\section{相关工作}
\subsection{相关工作一}
\subsection{相关工作二}
\subsection{相关工作三}
\subsection{相关工作四}
\subsection{相关工作五}

\section{参考资料}
\begin{enumerate}
    \item 
    
\end{enumerate}
\end{document}