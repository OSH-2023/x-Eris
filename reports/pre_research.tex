\documentclass[a4paper]{article}
\usepackage[text={165mm,245mm}]{geometry}
\usepackage{graphicx}
\usepackage{subfigure}
\usepackage{ctex}
\usepackage{float} 
\usepackage{listings}
\usepackage{xcolor}
\usepackage{amsmath}
\usepackage{hyperref}
\definecolor{mygreen}{rgb}{0,0.6,0}  
\definecolor{mygray}{rgb}{0.5,0.5,0.5}  
\definecolor{mymauve}{rgb}{0.58,0,0.82}  
  

\title{x-Eris 调研报告}
\author{小组成员:胡天羽,罗胤玻,李润时,万辰希,吴书让}
\begin{document}
\maketitle

\section{项目概述}
\section{项目背景}
\subsection{嵌入式操作系统}
\subsubsection{概述}
\subsubsection{小结}

\subsection{非虚拟文件系统}
\subsubsection{概述}
\subsubsection{小结}

\subsection{虚拟文件系统 - Linux}
\subsubsection{概述}
\subsubsection{小结}

\subsection{虚拟文件系统 - 嵌入式}
\subsubsection{概述}
本部分主要介绍了 \texttt{FreeRTOS-Plus-FAT} 以及 \texttt{RT-Thread} 中的文件系统实现细节。

在早期的嵌入式系统中,需要存储的数据比较少,数据类型也比较单一,往往使用直接在存储设备中的指定地址写入数据的方法来存储数据。然而随着嵌入式设备功能的发展,需要存储的数据越来越多,也越来越复杂,这时仍使用旧方法来存储并管理数据就变得非常繁琐困难。
因此需要新的数据管理方式来简化存储数据的组织形式,需要新的文件系统设计。

\subsubsection{\texttt{FreeRTOS-Plus—FAT}}
通过阅读源代码,可以简单分类此项目的文件组织如下。
\begin{itemize}
\item 主要
\begin{itemize}
    \item \texttt{ff\_dir}: 用于访问文件夹中内容
    \item \texttt{ff\_fat}: 用于访问 \texttt{FAT} 文件系统
    \item \texttt{ff\_file}: 用于文件读写
    \item \texttt{ff\_ioman}: 管理缓存和挂载读写对象(介质)
    \item \texttt{ff\_format}: 格式化或分区介质
    \item \texttt{ff\_locking}: 加锁?
    \item \texttt{ff\_memory}: 从内存读取数据
    \item \texttt{ff\_stdio}: 用于文件管理(统计),相对路径转换
    \item \texttt{ff\_sys}: 用于映射文件系统到根目录
\end{itemize}
\item 辅助
\begin{itemize}
    \item \texttt{ff\_headers}: 管理所有头文件
    \item \texttt{ff\_time}: 获取时间
    \item \texttt{ff\_error}:  用于错误处理
    \item \texttt{ff\_crc}: 用于计算 \texttt{CRC}(循环检验码)
    \item \texttt{ff\_string}: 字符串库
\end{itemize}
\item 驱动
\begin{itemize}
    \item 各处理器相应的文件系统驱动
\end{itemize}
\end{itemize}

\paragraph{\texttt{ERRNO} 值} \texttt{FreeRTOS-Plus-FAT} 文件系统的标准 API 与标准 C 库使用相同的 \texttt{errno} 值。
标准 C 库中的文件相关函数返回 0 表示通过,返回 -1 则表示失败。如果返回 -1,则失败的原因 存储在名为 \texttt{errno} 的变量中,须单独检查。 
同样,\texttt{FreeRTOS-Plus-FAT} 的标准 API 返回 0 表示通过,返回 -1 则表示失败, 该 \texttt{API} 还会针对各项 \texttt{RTOS} 任务维护 \texttt{errno} 变量。

\subsubsection{\texttt{RT-Thread}}
此文件系统有较多简介,可供设计参考,它的特点是:
\begin{itemize}
    \item 为应用程序提供统一的 \texttt{POSIX} 文件和目录操作接口:\texttt{read}、\texttt{write}、\texttt{poll/select} 等。
    \item 支持多种类型的文件系统,如 \texttt{FatFS}、\texttt{RomFS}、\texttt{DevFS} 等,并提供普通文件、设备文件、网络文件描述符的管理。
    \item 支持多种类型的存储设备,如 \texttt{SD Card}、\texttt{SPI Flash}、\texttt{Nand Flash} 等。
\end{itemize}
\paragraph{POSIX 文件系统接口}

\texttt{POSIX} 表示可移植操作系统接口,\texttt{POSIX} 标准定义了操作系统应该为应用程序提供的接口标准,是 \texttt{IEEE} 为要在各种 \texttt{UNIX} 操作系统上运行的软件而定义的一系列 \texttt{API} 标准的总称。

此标准意在期望获得源代码级别的软件可移植性。换句话说,为一个 \texttt{POSIX} 兼容的操作系统编写的程序,应该可以在任何其它 \texttt{POSIX} 操作系统(即使是来自另一个厂商)上编译执行。\texttt{RT-Thread} 支持 \texttt{POSIX} 标准接口,因此可以很方便的将 Linux/Unix 的程序移植到 RT-Thread 操作系统上。

\paragraph{虚拟文件系统层}
用户可以将具体的文件系统注册到 \texttt{DFS} 中,如 \texttt{FatFS}、\texttt{RomFS}、\texttt{DevFS} 等,具体介绍请参考其他小节。

\paragraph{设备抽象层}
设备抽象层将物理设备如 \texttt{SD Card}、\texttt{SPI Flash}、\texttt{Nand Flash},抽象成符合文件系统能够访问的设备,例如 \texttt{FAT} 文件系统要求存储设备必须是块设备类型。

不同文件系统类型是独立于存储设备驱动而实现的,因此把底层存储设备的驱动接口和文件系统对接起来之后,才可以正确地使用文件系统功能。
\subsubsection{小结}
设计 \texttt{VFS} 时,可以参考 \texttt{RT-Thread} 的三层结构组织形式、\texttt{FreeRTOS-Plus-FAT} 的API函数设计理念,将 \texttt{FreeRTOS-Plus-FAT} 继续拓展优化。

例如,可通过以下几点提高\textbf{兼容性}。(利用 \texttt{POSIX} 文件系统接口和与 \texttt{FreeRTOS-Plus-POSIX} 的结合将有可能方便地运行从 \texttt{Linux/Unix} 上移植的程序。
虽然目前 \texttt{FreeRTOS-Plus-POSIX} 对 \texttt{POSIX API} 的支持还不够充分,但是随着开发实用价值将逐渐上升。)

\begin{itemize}
    \item 继续采用标准 \texttt{eerno} 值
    \item 添加 \texttt{POSIX} 文件系统接口
    \item 添加支持的文件系统(磁盘文件系统、闪存文件系统)
    \item 拓展支持的实用函数
\end{itemize}

\subsection{文件系统开发}
\subsubsection{概述}
\subsubsection{小结}

\section{立项依据}

%\begin{figure}[H]
%    \centering
%    \includegraphics[scale=0.37]{p1.jpg}
%    \caption{图1}
%\end{figure}


\section{前瞻性/重要性分析}
\subsection{...}
\section{相关工作}
\subsection{\href{https://www.freertos.org/zh-cn-cmn-s/FreeRTOS-Plus/FreeRTOS_Plus_FAT/index.html}{FreeRTOS-Plus-FAT}}
是本项目的基础,主要使用 C语言 写成。源代码组织简洁,内容完整。但没有文档支持,建议直接阅读代码。
其主要可借鉴特性:
\begin{itemize}
    \item 采用标准 eerno 值
    \item 项目结构组织清晰
\end{itemize}
\subsection{\href{https://www.rt-thread.org/document/site/\#/rt-thread-version/rt-thread-standard/programming-manual/filesystem/filesystem}{RT-Thread/DFS}}
是RT-Thread的虚拟文件系统实现,文档详尽,可与上一个项目结合理解虚拟文件系统的一些设计理念。
其主要可借鉴特性:
\begin{itemize}
    \item 采用 POSIX 接口
    \item 目录管理相关使用函数较多
    \item 有关于文件系统配置方面参数的介绍
\end{itemize}

\subsection{相关工作二}
\subsection{相关工作三}
\subsection{相关工作四}
\subsection{相关工作五}

\section{参考资料}
\begin{enumerate}
    \item 
    
\end{enumerate}
\end{document}